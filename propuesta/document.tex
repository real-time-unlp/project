% !TeX spellcheck = es_ES
\documentclass[a4paper,11pt]{article}
\usepackage[spanish]{babel}
\usepackage[left=3cm,right=3cm]{geometry}
\usepackage[utf8]{inputenc}
\usepackage{times}
%opening
\title{Semáforos inteligentes
		\\{\large Propuesta de proyecto\\ Sistemas de tiempo real 2017}}

\author{García, Agustín Manuel
\\Romero Dapozo, Ramiro
\\Ternouski, Sebastian Nahuel}

\date{}

\begin{document}

\maketitle

La propuesta se basa en diseñar semáforos inteligentes capaces de detectar los autos que llegan a las intersecciones y, en función de eso, tomar una decisión de cuál es el que tiene prioridad de paso.

Adicionalmente, los semáforos se equipan con sensores de radiofrecuencia que permiten detectar ambulancias, policías y bomberos para darles mayor prioridad de paso.
Para ello los vehículos con mayor prioridad deben tener un emisor de radiofrecuencia para anunciar su cercanía a un semáforo inteligente. 

Cada semáforo tiene un microcontrolador que lo gobierna y se comunican entre sí mediante algún protocolo robusto y apto para sistemas de tiempo real como Can BUS.

La toma de decisiones se hace de forma conjunta: cada semáforo tiene información de los demás, y se comparan los resultados del análisis que haga cada semaforo, tomándose como definitiva la estrategia que haya sido elegida por la mayoría de ellos.

Por último, los peatones pueden apretar un botón, ubicado en cada semáforo que les permite generar un pedido de cruce.

\section*{Lista tentativa de materiales}
\begin{itemize}
	\item Arduino Uno o Mega
	\item Adaptadores de red CAN Bus y HW relacionado
	\item Sensores ultrasónicos
	\item Pulsadores
	\item Transmisores/receptores RF
\end{itemize}

\end{document}
